
\subsection{Random effects approach}

Figures~\ref{fig:ergmm_an} and~\ref{fig:ergmm_se} show the results of latent cluster random effects models for the earliest legislature of the National Assembly, during which the \textit{Front national} formed a parliamentary party group, and for the latest legislature of the Senate, in which Radicals and Greens both sit in their own party groups. The plots overlay cluster assignments to empirical party membership in order to illustrate the strengths and limitations of the model, which are both present in the shown series.%

$$\textrm{[FIGURES~\ref{fig:ergmm_an} AND~\ref{fig:ergmm_se}]}$$

The graphs show MPs placed by their scaled coordinates $Z_1$ and $Z_2$, or Minimum Kullback-Leibler (MKL) estimates, in the latent space $\bar Z$ \citep[p.~7]{KrivitskyHandcock2009-SN}. Because the estimates control for collective and individual differences in the amount of legislation produced, MPs with low cosponsorship activity are clustered with their more active cosponsors, instead of being represented as `hub and spoke' star subgraphs around them \citep[p.~10]{KrivitskyHandcock2009-SN}. We illustrate the difference by showing the model results along a force-directed representation of the same network, using the Fruchterman-Reingold algorithm \citep{FruchtermanReingold1991-SPE} implemented by \citet{Butts2008-JSS}.%

In the language of partisan veto players \citep{Tsebelis2002-PUP}, the latent cluster space shows party groups centered around their ideal points in a two-dimensional policy space, which offers four possible directions for party groups to distantiate themselves from each other. The model shows how smaller opposition party groups occupy this space by placing them differently around the ruling or opposition majority groups, as illustrated in Figure~\ref{fig:ergmm_se}.%

Despite some limitations in the cluster assignment procedure, as when the identification of a subgraph of isolated cosponsors causes all other unassigned MPs to join a single large residual group, the model successfully assigns large fractions of each parliamentary group to substantively meaningful clusters: for most parties, the models grouped roughly 80\% of members into a single cluster, and almost always assigns all remaining single-cluster outliers to the other leftwing or rightwing party cluster, as expected in a ``quadrille bipolaire'' two-party configuration characterised by high party cohesion \citep{Duverger1968-PUF,Tsebelis2002-PUP}.%

To summarize the overall quality of our latent estimates of party differentiation, we compare empirical party groups to latent cluster assignments in each legislature by computing the Fowlkes-Mallows index, a quality metric of their crosstabulation that converges to 1 as the quality of the hierarchical clustering increases \citep{FowlkesMallows1983-JASA}.%
  %
  \endnote{The Fowlkes-Mallows index is computed from $( \sum{T^2} - N ) / \sqrt{ (\sum{P^2} - N) \times (\sum{K^2} - N) }$, where $T$ is the crosstabulation of true party groups $p$ and latent cluster assignments $k$, $\sum P^2$ and $\sum K^2$ the squared sum of MPs in each of them, and $N$ the total number of MPs.} %
  %
  We compute the same index to assess the success of hierarchical clustering in each chamber, legislature and party group, as well as in every left-right coalition of party groups, by fusing leftwing and rightwing parties together prior to comparing them to latent clusters.%

The results, shown in Figure~\ref{fig:fm}, are moderately high in both chambers, very high in highly disciplined party groups, and generally higher in recent legislatures, for which the predictability of party affiliation from the latent space of cosponsorships has increased from approximately 0.6--0.7 to 0.8--0.9. From the patterns of their cosponsorships, the most predictable MPs are Communists, and the least predictable MPs are Senators from the Centrist (rightwing) and Radical (leftwing) party groups, which is consistent with the longer history of these party formations \citep[p.~70-71 and 77-78]{Sauger2010}.%

$$\textrm{[FIGURE~\ref{fig:fm}]}$$

The results also show a clear observable difference in the overall predictability of left-right coalition alignment in each chamber: in aggregate, two random MPs from the same leftwing or rightwing party coalition had a systematically higher probability to be assigned to the same cluster in the National Assembly than in the Senate, where the predictability of left-right alignment is only moderately high by that metric. This difference persists even if Radical Senators, who include both leftwing and rightwing MPs, are removed from the computation \citep[as suggested in][p.~80]{Sauger2010}.%

\subsection{Fixed effects approach}

We further explore left-right alignment and party cohesion from a fixed effects perspective. Figure~\ref{fig:ergm_beta} shows ERGM coefficients for the `full' model detailed in Section~\ref{sec:ergm} (p.~\pageref{sec:ergm}), along with the coefficients of a `baseline' model that includes all predictors except the within-party differential homophily term, for which coefficients are shown separately in Figure~\ref{fig:ergm_diff}. Standard errors and Bayesian Information Criteria (BIC) are reported in Tables~\ref{tbl:ergm_an} and ~\ref{tbl:ergm_se} at the end of this section.%

$$\textrm{[FIGURES~\ref{fig:ergm_beta} and~\ref{fig:ergm_diff}]}$$

Three series of estimates in Figure~\ref{fig:ergm_beta} show signs of strong dyadic dependence in the cosponsorship networks. The first row simply shows intercepts for network edge counts. The second row shows the high influence of reciprocality in cosponsorship, especially among Senators \citep[see also][p.~78]{CranmerDesmarais2011-PA}. The last row of coefficients further shows that, in the `baseline' (party-free) model, Senators are generally as likely to be mutual cosponsors as they are \emph{unlikely} to come from different leftwing and rightwing coalitions. The same relationship exists for the National Assembly, but to a lesser extent that leaves more influence to left-right alignment than to mutuality.%

The remaining rows of ERGM coefficients show that similarity in gender and seniority generally have very little influence on the formation of cosponsor dyads. All estimates for the latter are insignicant effects ranging near zero, whereas gender homophily has a single remarkable positive effect in legislature~11 (1997-2002) of the National Assembly. For this chamber and legislature, gender homophily is estimated to be predictive of cosponsorship within the same order of magnitude than reciprocality, and is robust to controlling for within-party homophily. This effect might be traceable to bill-specific events, such as welfare and women's issues \citep{ClarkCaro2013-PG}, but our bill-level data are too limited to explore that hypothesis fully.%

Figure~\ref{fig:ergm_diff} finally shows the differential homophily coefficients for major party groups, which measure the strength of intra-party ties after controlling for the previous covariates. In the previous series of results, the strongest node-level effects designated left-right differentiation as the strongest (negative) predictor of cosponsorship. In this part of the model, the complementary (positive) effect of party loyalty is estimated on a range of magnitude that only slightly exceeds that of all other estimates, from approximately 0 to 5 log-odds, with most effects between 0 and 2. Taken overall, the coefficients show either weak effects of party homophily in early legislatures, or stronger effects in recent ones, along with larger standard errors.

Some estimation issues appear in the ERGM of legislature~13 (2007-2012), for which most coefficients have very large standard errors. This limitation precludes any firm conclusion on the precise variation of within-party homophily in recent years. Our estimates therefore confirm that party cohesion is very high in both chambers, but only suggest that it is currently on the rise: within-party homophily among Conservatives, for instance, is currently exceptionally high in the National Assembly, but this level of intra-party loyalty might be accounted for by intensive obstruction strategies against highly politicized government bills like gay marriage, which generated an exceptionally high volume of (mostly single-authored) obstructive amendments, and might fade out later on.%

Further statistical significance issues affect the small Communist and Socialist party groups in early legislatures of the National Assembly, in which very small numbers of authored bills cause the ERGM coefficients to switch to negative values. The issue affects legislature~9, for which there are only seven Socialist-authored bills in the network sample, and legislature~11, during which Socialist MPs led a three-party leftwing coalition and again contributed only a very small fraction of all MP-authored bills (around 5\%). These issues are thus both explainable from the composition of the network sample but might still be different in nature, insofar as the odd result of legislature~11 might be substantively explainable.%

With these limitations in mind, the results still provide sufficient evidence to confirm the presence of very high party homophily, coupled to high levels of cosponsor mutuality and left-right differentiation in the observed networks. The unlikelihood of left-right legislative collaboration is higher in the National Assembly, which is consistent with the previous finding that left-right coalitions in this chamber were also relatively more successfully clustered from the latent space model covered. Finally, party homophily is generally similar in both chambers, and possibly higher in the recent period.%

$$\textrm{[TABLES~\ref{tbl:ergm_an} AND~\ref{tbl:ergm_se}]}$$
