
Our analysis aimed at examining party polarization in the French Parliament from a network approach, using a set of fourteen legislature-level networks that cover both chambers since 1986. As a firsthand approach to the data, we started by modeling random effects and latent clusters in each cosponsorship network, passing party affiliation and all other covariates to a two-dimensional latent space that captured both left-right and party differentiation.%
  %
  \endnote{We might call this world `Flatland', to make its nature clearer to those who are privileged to live in Space \citep[p.~35]{Abbott2009-BP}.} %
  %
  We then estimated left-right and party differentiation parameters as fixed effects, controlling for network structure and social covariates.%

Consistent with studies of party discipline from roll-call votes \citep{GodboutFoucault2013-FP}, our results confirm very high levels of party discipline under a variety of parliamentary configurations that all express a strong level of left-right differentiations between party groups. The variations that we found in the networks are also generally consistent with other research that underline the higher volatility of Centrist and Radical MPs, and the higher discipline of far-left party groups.%

Our results also underline the need for more exhaustive legislative data in machine readable format \citep[p.~82]{Sauger2010}. Although we used a quasi-exhaustive sample of all cosponsored legislation available from the National Assembly and Senate websites, the scope of parsable legislation before 2002 is limited to bills, which in turn limits the identifiability of weak collaborative signals like cosponsorship ties, especially for smaller party groups.%
