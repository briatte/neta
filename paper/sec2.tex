% data

Although formally intended to encourage technical collaboration between branches of government over lawmaking \citep{Heller2001-AJPS}, legislation also translates the struggle of government branches over the control of the legislative agenda. In several studies of U.S. legislatures, this struggle has been shown to form regular patterns of collaboration between legislators, the structure of which can help predict legislative productivity \citep{Fowler2006-PA,ChoFowler2010-JP}.%

Members of the French Parliament are provided limited resources and enjoy only restricted control over the legislative agenda in comparison to the executive \citep{Kerrouche2006-JLS,BaumgartnerBrouard2014-G}. In that context, many bills introduced by MPs represent attempts by opposition parties to express defiance at the government and obstruct its legislative drive \citep{Conley2011-FP}. Similarly, a large volume of amendments produced during parliamentary plenary debates serve to grant their authors a few minutes of speaking time during the passage of a bill.%

Legislation therefore plays a role in framing parliamentarians as veto players in the policy process \citep{Tsebelis1999-APSR}. For that reason, as a previous study of the French National Assembly has observed, ``the volume of private members' bills, especially those from minority deputies, may be regarded as a reasonable proxy for the level of conflict between the government and opposition deputies. It will also reveal something of the extent of the government’s political control over its backbenchers'' \citep[p.~343]{Kerrouche2006-JLS}.%

\subsection{Sample definition}

In this study, we focus on the relational determinants of legislative behaviour, and measure legislative collaboration with the aim to control, rather than to account, for the volume of legislation passed. Our sampling frame thus consisted in all available legislation for which one MP could be nominally identified as the author, and one or more other MP(s) could be nominally identified as cosponsor(s). This design concentrates on ties between first authors and their cosponsors within each chamber, as does \citet{Fowler2006-PA} on similar data.%

The legislation collected for this research consist of slightly over 103,000 items sponsored over the past 27 years, from 1986 to today. The 96,168 amendments and 6,864 bills used to build the cosponsorship networks presented in the next sections are distributed over seven legislatures, or roughly half of the Fifth Republic, and come from a larger sample approximately twice that size, which also includes single-sponsored MP legislation, government bills and nonbinding resolutions.%

The data were produced by scraping sponsor details from the National Assembly (\url{http://www.assemblee-nationale.fr/}) and Senate (\url{http://www.senat.fr/}) websites. Legislation data for the National Assembly were retrieved from its online pages, and legislation data for the Senate were retrieved from the SQL database dumps of its open data initiative (\url{http://data.senat.fr/}). Less than 2\% of the data could not be processed due to missing information or parsing issues.%

Figures~\ref{fig:counts_an} and~\ref{fig:counts_se} show the time distribution of the data over each National Assembly term (\emph{législature}), which coincide with executive terms from 2002 onwards. Legislatures~8 (1986-1988), 10 (1993-1997) and 11 (1997-2002) include periods of divided government \citep{BaumgartnerBrouard2014-G}, with three leftwing parties in government during legislature~11; all other periods represent periods of government under leftwing or rightwing two-party coalitions.%
  % \endnote{See \citet[p.~313]{GodboutFoucault2013-FP} for more exhaustive information on parliamentary legislatures in the Fourth and Fifth Republics. The data collected for this study contain only basic MP socio-demographics; for a more detailed examination, see \citet{FrancoisGrossman2011-I}.}%

$$\textrm{[FIGURE~\ref{fig:counts}]}$$

Table~\ref{tbl:counts} provides further information on the number of MPs and parliamentary party groups represented in the cosponsorship data. The number of MPs per legislature is sometimes superior to the number of seats due to replacements and delays in legislation that allowed sponsors from multiple legislatures to cosponsor the same bill. The number of groups indicates how many parties have more than 9 members in the cosponsorship sample, which is the lowest threshold to form a party group in (Senate) parliamentary rules.%
  %
  \endnote{The threshold to form a party group in the National Assembly was lowered from 20 to 15 members in 2009. Our figure futher ignores unaffiliated MPs.}%

$$\textrm{[TABLE~\ref{tbl:counts}]}$$

The most serious limitation in the data explains the highly unequal number of items before and after 2002  in both chambers: amendments data were available only since 2004 in the National Assembly or since 2001 in the Senate, and do not include amendments submitted to specialised committees, where a lot of ``invisible'' legislative collaboration occurs \citep[p.~357]{Kerrouche2006-JLS}. Another major issue specifically affects legislature~10 in the National Assembly bills series, which is composed of only a few cosponsored bills at the end of the legislature. For both reasons, we first ran our analysis on amendments and bills separately, and then on the combined series.%

\subsection{Network construction}

Cosponsorship networks were built out of the ties formed between the first author of a bill amendment, and all other sponsors of the proposed legislation. This definition of network ties matches the criteria used in network analyses of cosponsorship in the U.S.~Congress \citep{Fowler2006-PA,GrossShalizi2012}, and relies on a similar constructor, namely a two-mode edge list of the form%

 \[ \begin{array}{cccccc}
    	\{ a_8 , b_1 \} , & \{ a_{31}, b_1 \} , & \{ a_{27}, b_2 \} , & \dots & & \\
    	\vdots & & & & & \\
      & & & \dots & \{ a_{36}, b_{n-1} \} , & \{ a_{120}, b_n \}
    \end{array} \]

with MP sponsors denoted $a_i$ and legislation items denoted $b_k$, regardless of their kind. The construct under examination is therefore an initially bipartite, or two-mode, network of $a \times b$ ties between MPs and legislation items. To focus the study on collaboration between legislators, we collapse this construct to a one-mode network containing strictly MPs, by connecting the first author of each item to all other sponsors of the item. The resulting adjacency matrix $A$ of directed ties between MPs $(i,j)$ is an asymmetric matrix with elements%

    \[ A(i,j) = \left\{
      \begin{array}{l l}
        1 & \quad \text{if MP $j$ cosponsored an amendment or bill by MP $i$,}\\
        0 & \quad \text{otherwise.}
      \end{array} \right. \]

and where all diagonal elements describing network self-loops --- MPs hypothetically cosponsoring legislation with themselves --- are discarded.%

%  The network can be described as a directed edge list of the form%
%
%  \[ \begin{array}{cccccc}
%       \{ i_8 , j_{15} \} , & \{ i_{8}, j_{190} \} , & \{ i_{31}, j_2 \} , & \dots & & \\
%       \vdots & & & & & \\
%       & & & \dots & \{ i_{120}, j_{43} \} , & \{ i_{120}, j_{48} \}
%     \end{array} \]

Since cosponsorship between MPs $i$ and $j$ can occur more than once during a legislature, the ties of their network must then be valued to reflect their different strength. To do so, we followed \citet[Equation.~1]{GrossShalizi2012} by weighting all cosponsorships in inverse proportion to the overall number of cosponsors on the item, and by normalizing their sum to the maximum number of possible cosponsorships between MPs $i$ and $j$.%
  %
  \endnote{The resulting weights are bounded between $0$ and $1$ and are approximately log-normal in several of the collected networks.}%

Finally, for the purpose of this study, we recoded a myriad of parliamentary party group affiliations to a simplified list of seven leftwing and rightwing political families. Our recoding is broadly similar to that of \citet[p.~310]{GodboutFoucault2013-FP}: for instance, we also coded the Red-Green leftwing `GDR' coalition of legislature~13 (2007-2012) as a single group. This classification offers a spectrum of four leftwing and three rightwing party groups: Communists/Red-Greens, Greens, Socialists and Radicals on the leftwing, and Centrists, Conservatives and the \emph{Front national} on the rightwing.%
  %
  \endnote{The \emph{Front national} formed a single party group in legislature~8 of the National Assembly (1986-1988).}%

Three slight differences exist between our classification and the one used by \citet[p.~310]{GodboutFoucault2013-FP}. First, in order to better capture leftwing party divisions (especially in the Senate), we did not collapse Radical MPs with Socialists MPs when both formed separate parliamentary groups. We also decided to code \emph{Démocratie libérale} (DL) National Assembly MPs as rightwing Conservatives (RPR/UMP) rather than Centrists. Last, we left all unaffiliated MPs (\emph{non inscrits} and \emph{sans étiquette}) out of our classification. These differences create nontrivial differences in party codes between both classifications, but do not seriously threaten the overall comparability of the samples used.%

The replication material for this paper, which was written in \texttt{R}~\citep{R}, contains all necessary functions to download the raw data and process it to weighted network objects, which can be easily visualized using several placement algorithms \citep{Butts2008-JSS} or converted to other formats. The replication material contains the postprocessed legislation data up to April 6, 2014, which we used in the present analysis. %
   \endnote{See \url{https://github.com/briatte/neta}. The replication material can update the data past its current endpoint, so that legislature~14, which started in 2012, can be supplemented with more information as it becomes available.}%
