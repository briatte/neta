% intro

Previous studies of (co)sponsored legislation have found it to be a useful heuristic to how Members of Parliament (MPs) signal positions to other legislators \citep{KesslerKrehbiel1996-APSR}, to the executive, or to constituencies such as interest groups and voters \citep{WilsonYoung1997-LSQ}. This kind of position taking is further encouraged by the weakly constrained nature of (co)sponsorship, over which legislators generally enjoy more control than they do over votes \citep{Schiller1995-AJPS}.%

A large fraction of existing research on legislative networks is based on U.S. legislative chambers, for which there are large scale historical data available \citep{ZhangFriend2008-P,ClarkOsborn2009-SPPQ}. Drawing extensively from this literature, this paper examines the extent of party polarization expressed by members of the French Parliament through their propensity to cosponsor legislation together, as an alternative  measurement strategy to roll-call voting records \citep{Sauger2010}.%

Section~\ref{sec:data} starts by introducing original network data built from around 100,000 cosponsored amendments and bills introduced in both chambers of the French Parliament over the past 27 years. Section~\ref{sec:methods} then presents a methodological framework that relies on exponential random graph models \citep{KrivitskyHandcock2009-SN,CranmerDesmarais2011-PA} to measure the influence of party identification on cosponsorship in each of the seven parliamentary legislatures since 1986.%

In accordance with roll-call spatial voting models of French legislative politics \citep{GodboutFoucault2013-FP}, the results presented in Section~\ref{sec:results} tend to reveal a primarily one-dimensional policy space structured by left-right positions and strong party loyalty. Both random and fixed effects models of the cosponsorship data find moderate to strong party-based network patterns across all parliamentary groups that are largely consistent with longer historical patterns of party unity in postwar France \citep{Sauger2010}.%

The overall results of our study, which we further discuss in Section~\ref{sec:discussion}, show that party polarization has increased in recent legislatures, as Members of Parliament sponsor more and more legislation within party lines. The two-by-two multi-party configuration popularized decades ago by \citet{Duverger1968-PUF} under the name ``quadrille bipolaire'' is still empirically relevant to qualify both past and present left-right coalitions in Parliament, and is highly predictive of legislative interactions conducted under the control of highly disciplined party coalitions.%
