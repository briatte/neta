The cosponsorship networks under study are large-scale objects built out of primarily weak signals between individual actors (MPs) who carry only limited information with regards to the overall topology of legislative collaboration occuring in the network as a whole. Assessing the extent of cohesion or divisiveness expressed at the level of parliamentary party groups therefore requires paying attention to the structural properties of cosponsorship networks along other covariates \citep{KirklandGross2012-SN}.%

For this reason, we focus our analysis on characterizing each legislature through network-level properties, rather than through ego-level measures of influence such as rankings of MPs by legislative productivity. In order to preserve some of the benefits of exploring egocentric networks \citep[see, e.g.]{Box-SteffensmeierChristenson2014-SN}, we offer an interactive visualization of the cosponsorship network data, available online at \url{http://briatte.org/sigma}, that shows force-directed graph representations of the fourteen networks under study.%

We start by measuring the overall level of party polarization in the cosponsorship network of each legislature, finding variations in time as well as between chambers that are consistent with previous studies of party cohesion in the French Parliament. We then explain how we estimated party effects through exponential random graph models, using both random and fixed effects approaches \citep{KrivitskyHandcock2009-SN,CranmerDesmarais2011-PA}.%

\subsection{Network measures}

Figure~\ref{fig:measures} shows some weighted network properties for the fourteen legislature networks present in the data. The top row of graphs shows counts of nodes and ties in each networks, with a clear difference in network density when large volumes of amendments data are available in the series. The middle row shows weighted graph-level measures of centralization, distance and clustering in the networks \citep{OpsahlPanzarasa2009-SN,OpsahlAgneessens2010-SN}, which underline the higher intensity of cosponsorship activity in the National Assembly, in which cosponsors are more connected, more clustered and more numerous overall.%

$$\textrm{[FIGURE~\ref{fig:measures}]}$$

% Several of our network measures are expected to behave as mere covariates to the amount of available network data. In recent legislatures, for example, changes in centrality indices like betweenness \citep{Brandes2001-JMS} correlate with the inclusion of amendments data, and is not expected to reflect substantive changes in parliamentary collaboration. However, in the same legislatures, higher numbers of distinct parliamentary groups also translate into more opportunities for MPs to associate with cosponsors outside of their own party, which results in higher network-level opportunities for brokerage across structural holes, and therefore lower node-level network constraint \citep{Burt2004-AJS}.%

The bottom row of network measures provides information on network modularity, which compares the proportion of network ties formed within a given vector of groups to that same proportion in a randomized network of identical dimensions. The further away the observed network is from the randomized `null' model, the more efficient the group vector is at partitioning the network into meaningful communities \citep{Newman2006-PR}. In a party-based setting, modularity therefore measures the level of within-party collaboration against random collaborative ties, and serves as a proxy for party polarization expressed in network data \citep{ZhangFriend2008-P,WaughPei2009-A,Kirkland2013-SPPQ}.%

The methodology to compute modularity is retraced in \citet{WaughPei2009-A} and \citet{KirklandGross2012-SN}. For a given network partition membership variable $g$, which in this case is the parliamentary party group affiliation of each MP, the modularity $M$ represents the fraction of weighted network degree $m$ contained inside the community $g$, minus the expected total degree of all cosponsorships in the network:

     \[ M = \frac{1}{2m}\displaystyle\sum_{i,j=1}[ A(i,j) - P(i,j) ]\delta(g_i,g_j) \]

     where $m = \frac{1}{2} \sum_i k_i$ is the total number of cosponsorships in the network, $P(i,j)$ is the expected total degree of the network, $g_i$ is the community to which a given MP belongs, and $\delta(g_i, g_j) = 1$ if the MP $i$ and her cosponsor $j$ belong to the same community, or $0$ if they do not. The randomization component used in the null model preserves the weighted degree distribution of the network by being equal to $P(i,j) = \frac{k_i k_j}{2m}$, where $k_i = \displaystyle\sum_{j=1} A(i,j)$ is the weighted sum of cosponsorships per MP \citep{Newman2006-PR}.%

We further followed \citet[Section~3]{WaughPei2009-A} into maximizing our modularity estimates. We first obtained two alternative membership vectors for the observed networks from the Louvain \citep{BlondelGuillaume2008-JSMTE} and Walktrap \citep{PonsLatapy2006-JGAA} algorithms, which identified optimal partitions through multilevel classification and through 1--50 random walks respectively. We then compared empirical party-based modularity with the highest modularity score of these `maximized' partitions, $\max M$, by taking their ratio. All three measures are plotted in the bottom row of Figure~\ref{fig:measures} and shown in Table~\ref{tbl:modularity}.%

$$\textrm{[TABLE~\ref{tbl:modularity}]}$$

Both empirical and maximized measures of modularity are, by our estimates, indicative of much higher party polarization in recent legislatures, possibly since legislature~10 (1993-1997), but most visibly and increasingly so since legislature~12 (2002-2007), following the last period of divided government. This characteristic of the network data is consistent with high measures of party unity in roll-call votes during the same period \citep[p.~320]{GodboutFoucault2013-FP}, and also conforms to expectations driven by the presence of highly amended legislation in recent decades, such as finance and social security laws.

% Maximizing network modularity further informs us on the degree of party-based fragmentation by relaxing the empirical number of party groups in the network and identifying a higher, `optimized' number of groups, or communities, at which modularity is maximized. Differences between empirical and maximized numbers of communities in each legislature, which are shown in Table~\ref{tbl:modularity}, indicate that the Louvain algorithm almost always identified less communities than the Walktrap algorithm, and that the Senate networks always required fewer communities than National Assembly networks to reach maximization.%

We finally learn from the measure of party-based modularity in the observed networks that Senate networks express a higher degree of party differentiation, which translates into more inter-party collaboration being observable in the National Assembly. This measure, however, does not separate variation within and outside of the left-right cleavage that structures party coalitions \citep[p.~72-74]{Sauger2010}. and guides the modeling strategy outlined below, which aims at decomposing left-right and party differentiation into more specific probabilities.%

\subsection{Network models}

In a simplified view that overlooks only small numbers of MPs, French parliamentary politics over the observed period are configured as quadrangular conflicts between a ruling two-party rightwing or leftwing majority and a two-party opposition, modulo one additional minority party when leftwing majorities include Green MPs. This representation of the policy space translates the original insight of the ``quadrille bipolaire'' \citep{Duverger1968-PUF}, from which we derive three possible types of hypothetical interactions within the cosponsorship networks:%

\begin{description}

  \item[H1] \emph{Partisan cohesion} might be measured as the level of intra-group collaboration, when legislation is produced exclusively or quasi-exclusively within party lines. In this case, the cosponsorship networks will express strong homophily, with ties being much more likely to occur between MPs of a same party, and especially so when party polarization in the chamber is high.%

  \item[H2] \emph{Left-right distance} might be measured as the (negative) likelihood of collaboration between rightwing and leftwing MPs, as opposed to the likelihood of collaboration between two MPs of either majority. Since we assume high levels of party polarization overall, we expect left-right distance to be systematically present in all parliamentary configurations under study.%

  \item[H3] \emph{Social covariates} might finally explain further associative or dissociative behaviour in the networks. Past research in legislative collaboration has found higher likelihood of MPs to cosponsor legislation by MPs who are in similar in sex or seniority to them \citep{BrattonRouse2011-LSQ,ClarkCaro2013-PG}, and we also expect some reciprocality between pairs of mutual cosponsors.%

\end{description}

%%% Working from these background assumptions about the directions of the relationships at play in discouraging or favorizing the bulk of cosponsorship activity between MPs, we ran two series of exponential random graph models (ERGMs), a modeling technique that ``allows researchers to model observed network structures as a function of actor-level variables, dyadic variables (such as political distance), and higher-order network effects that are hypothesized to influence the formation of collaborative ties'' \citep[p.~606]{GerberHenry2013-AJPS}. For an extensive review of network inference from ERGMs, see \citet{CranmerDesmarais2011-PA}, who discuss a case study of similar cosponsorship ties in the U.S. Congress.%

Our first series of ERGMs takes a `random effects' approach to party differentiation by recovering information about party membership from a latent space model. The second series takes a `fixed effects' approach by measuring within-party homophily and left-right distance after controlling for graph-level (network) structure and node-level (MP) covariates.%

\begin{description}

  \item[Random effects model] %
  %
  We start by estimating a series of latent cluster random effects models  \citep{KrivitskyHandcock2009-SN}, which estimate a two-dimensional Euclidean space containing as many latent clusters as there are party groups in the observed legislature network. This latent space serves a hypothetical two-dimensional policy space parametered to match the number of cosponsorships, MPs and party groups in the legislature, which assumes that each chamber might act as a veto player, as might each party group within it \citep{Tsebelis1999-APSR}.%

  The model identifies the latent space position of each MP $i$ by approximating the count distribution of her cosponsorship ties $Y_{ij}$ with all other MPs $j$ as $\mu_{ij} \sim \textrm{Poisson}(\mu_{ij})$, controlling for overall network size and for the random propensity of each MP $i$ and $j$ to cosponsor legislation. The model therefore amounts to maximizing%

  \[ log(\mu_{ij})=\beta_0 - \|\bar Z_i - \bar Z_j\| + \delta_i + \gamma_j \]%

  where $\bar Z$ denotes the latent positions of MPs, and $\delta_i$ and $\gamma_j$ are random effect terms that capture the propensity of each MP to form either send or receive cosponsorship ties \citep[Equation~4]{KrivitskyHandcock2009-SN}. This design aims at assigning MPs to latent clusters based on similarities in their cosponsorship records, controlling only for variance in legislative activity per cosponsor.%

  The model, which comes from \citet{HoffRaftery2002-JASA} and \citet{Hoff2003-NAP} for the suggestion to add random effects, is implemented with the \texttt{latentnet} package \citep{KrivitskyHandcock2008-JSS}. The Markov chain Monte Carlo (MCMC) estimation used a chain burn-in of 100,000 iterations, an MCMC sample size of 5,000 and an interval between successive samples of 10 iterations.%

  \item[Fixed effects model] %
  \label{sec:ergm}%
  %
  In the second stage of our analysis, we fit each cosponsorship network to a fixed effects ERGM that estimates the likelihood of cosponsorship against both left-right and party differences, controlling for network size, cosponsorship reciprocality and similarity attraction by gender or seniority. The dependent variable is the probability of a cosponsorship tie $Y_{ij}$ to exist between each pair of MPs $i$ and $j$ in the network, which the model expresses conditionally to other specified dyadic independence terms, as in%

  \[
  P( Y_{ij} = 1 | n, Y_{ij}^{c} ) =
  \textrm{logit}^{-1}
  \begin{pmatrix}
    \displaystyle\sum_{k=1}^{K} \theta_{k} \delta ( \Gamma_{yk} )
  \end{pmatrix}
  \]

  where $n$ is the total number of MPs in the network, $Y_{ij}^{c}$ denotes all pairs of cosponsors other than $Y_{ij}$, $\Gamma_{yk}$ is a set of $K$ network statistics expected to differentiate the observed network from other possible networks of similar dimensions where $Y$ is randomized, and $\theta$ are the parameters estimated to maximize the likelihood of observing the network \citep[cited from][p.~71 and Equation~5]{CranmerDesmarais2011-PA}.%

  Within-party homophily is estimated from a differential homophily term \citep{KrivitskyHandcock2008-JSS} that captures the (expectedly positive) variation in likelihood of cosponsorship when both MPs come from that party. The (expectedly negative) likelihood of cosponsorship between two cosponsors coming from opposing left-right majorities is estimated from a network dyad dummy that codes for an absolute categorical difference of left-right affiliation in each pair of cosponsors.%
  %
    \endnote{Rightwing groups are Conservatives, Centrists and \emph{Front national} MPs. Leftwing groups are Communists, Greens, Radicals and Socialists.}%

  The model includes similar statistics to control for the likelihood of cosponsors to be similar in sex or seniority. We also follow \citet{CranmerDesmarais2011-PA} in adding a control for mutuality in the network structure, in order to avoid inflating within-party homophily coefficients with reciprocal cosponsorships. Stronger network structure effects were too limited in most observed networks to require passing additional controls for cyclicality and transitivity.%
    %
    \endnote{Furthermore, unlike \citet[p.~78]{CranmerDesmarais2011-PA}, the party homophily coefficients of our model did not turn insignificant when we experimented with adding triangle terms into the model equation for legislatures with sufficient transitiveties.}.%

  We further follow \citet[p.~78]{CranmerDesmarais2011-PA} in designing a parameter to subsample the network data prior to modeling it. Because our cosponsorship data are much sparser than they are in the U.S. Congress \citep{Fowler2006-PA}, the role of the threshold parameter in our design is less to `thin' the network than to operate as a sensitivity test for the ERGM coefficients, by regaining the information on edge weights lost in passing a dichotomized projection of the cosponsorship ties to the model. This approach led us to run four parallel series of constrained and unconstrained models along the results that we report, which affected the standard errors of our results without affecting coefficient signs.%
  %
  \endnote{Given that the distribution of edge weights was approximately log-normal in several networks, we used quantiles of the logged edge weights to parameter a default 95\% interval that dropped all edges weighted outside the 2.5\% and 97.5\% percentiles. We then ran the model on subsamples with no upper bound, no lower bound, on the full data and on a harsher sample that dropped up to 10\% of outlying network ties.}%

  The model is implemented by MCMC estimation through the \texttt{ergm} package \citep{HunterHandcock2008-JSS}, with a chain burn-in of 10,000 iterations, an MCMC sample size of 10,000 and an interval of 10 iterations between MCMC draws.%

\end{description}
